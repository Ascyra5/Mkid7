\section*{Wie wir Dinge auswerten können}

\subsection*{Beobachten und Messen}
In den letzten beiden Stunden haben wir uns viel mit Beobachtungen und Messungen beschäftigt.
Formuliere nochmal in eigenen Worten, was der Unterschied zwischen einer Beobachtung und einer Messung war. \\


\writeLines{4}{\textwidth}

Wennn wir uns an die letzten beiden Stunden erinnern - was genau haben wir beobachtet, und was genau haben wir gemessen? \\

\writeLines{4}{\textwidth}

\newpage

Nun versuchen wir das, was wir beobachtet haben mal auszuwerten. Was genau heißt das denn? Wozu benötigen wir eine Auswertung? \\

\writeLines{4}{\textwidth}

Nun versuchen wir mal mithilfe einer Auswertung folgende Fragen zu beantworten:

\subsubsection{Wie viel Volumen Wasser beinhaltet das  Glas bei einer Füllhöhe von 0,25 cm?}
\subsubsection{Was für eine Füllhöhe erreicht das Glas, wenn es 20 ml Wasser beinhaltet?}
\subsubsection{Was für eine Kantenlänge muss ein würfelförmiges Gefäß haben, dammit es bei einer Füllhöhe von 2 cm 54 ml Wasser beinhalten kann?}
\subsubsection{Bonus: Kannst du einen allemeinen Zusammenhang zwischen Kantenlänge, Höhe und Volumen herstellen?}

Zum Abschluss wollen wir nochmal alles Revue passieren lassen, was wir in den letzten Stunden gemacht haben. \\
Schreibe einen kleinen Text, mit dem du mithilfe der letzten Stunden erklärst, wie man in der Naturwissenschaft vorgeht, um wirklich etwas herauszufinden.
Was müssen wir zuerst tun? Wie können wir Dinge untersuchen? Wann und warum benutzen wir Mathe?