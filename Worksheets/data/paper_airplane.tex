\section*{Werkstatt}

    \subsection*{Wie arbeiten Wissenschaftler?}

        Erinnert ihr euch noch an das Tafelbild von letzter Stunde?
        Da hatten wir besprochen, was alles dazu gehötrt, wenn man etwas wissenschaftlich untersucht. \\
        Heute wollen wir fast den gesamten Weg selbst gehen!
        Damit aber keiner von uns irgendwo falsch abbiegt, schreibe dir nochmal auf, was alles dazu gehört, wenn man wissenschaftlich arbeitet.\\
        Schreibe dir auch auf, falls du Fragen hast, denn für die kommenden Projekte ist das wirklich wichtig :-) \\

        \karoBox{\textwidth}{12cm}

    \newpage

    \subsection*{Was machen wir heute?}

        Ich will hier gar nicht viel vorweg nehmen, denn gleich sollt ihr soviel wie möglich selber zu Wissenschaftlern werden. \\
        Letztes Mal haben wir spontan eine Papierfliegerwerkstatt aufgemacht. \\
        Derjenige, der mir den besten Papierflieger liefert, bekommt einen Preis!
        Und wenn ich sage, der beste - dann mein ich das so! Das bedeutet, legt euch ins Zeug, mich davon zu überzeugen, dass euer Flieger der beste ist, und besser, als die anderen!

        \subsubsection{Material:}

            \begin{itemize}
                \item Pro Flieger \textit{genau} 1 DIN A 4 Blatt
                \item Stifte zum \textit{verzieren}
            \end{itemize}

        \subsubsection{Regeln:}

            \begin{itemize}
                \item Es darf nur das vorgegebene Papier verwendet werden
                \item Das Papier darf nicht abgeschnitten werden, verkleinert oder vergrößert werden - es muss immer ein DIN A 4 Blatt bleiben
                \item Einreißen ist in Ordnung, abreißen nicht
                \item Es darf nichts auf den Flieger geklebt werden oder an den Flieger dran gehängt werden
            \end{itemize}

    \newpage

\section{Ziel und Thesen}

    \subsection{Worum geht es? Was wollen wir erreichen?}

            \writeLines{2}{\textwidth}

    \subsection{Stelle Vermutungen auf, von was die Flugweite deines Fliegers  abhängt. Das können und sollen mehrere Dinge sein.}

            \writeLines{4}{\textwidth}

    \subsection{Schreibe dir zu jeder Vermutung auf, inwiefern sie dein Flugzeug beeinflusst.}

            \writeLines{4}{\textwidth}


\section{Experiment planen}


    \subsection{Formuliere so präzise wie möglich deine Vermutung, die du testen möchtest.}

            \writeLines{2}{\textwidth}

    \subsection{Überlege dir nun, wie du am besten die vermutung zeigen (oder aber auch widerlegen) kannst. In anderen Worten: Überlege dir ein \textit{überzeugendes} Experiment, um deine Vermutung zu testen.}

            \karoBox{\textwidth}{8cm}

\section{Experiment durchführen}

    \subsection{Führe nun dein Experiment durch. Vergiss nicht, dir Messdaten aufzuschreiben - die benötigst du, um überzeugend zu sein!}

        \karoBox{\textwidth}{12cm}

    \subsection{Bestätigt dein Experiment deine Vermutung oder nicht?}

        \writeLines{2}{\textwidth}

\section{Auswertung}

    \subsection{Nimmst du nun all deine Ergebnisse zusammen - wie müsste laut deiner wissenschaftlichen Arbeit ein guter Papierflieger aussehen? Beschreibe ihn so genau wie möglich.}

        \karoBox{\textwidth}{14cm}

        \newpage

\section{Anwendung}

    \subsection{Baue nun deinen besten Papierflieger nach deiner wissenschaftlichen Arbeit.}

    \subsection{Wieso ist dein Flieger der beste? Versuche das so gut und überzeugend wie möglich durch deine Erkenntnisse zu begründen!}

    \subsection{Gestalte ein Werbeplakat, welches dein Flieger aufgrund deiner Daten als den bestmöglichen auszeichnet!}












