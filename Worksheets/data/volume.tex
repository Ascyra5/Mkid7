\section*{Wie wir Dinge vermessen können}

Letzte Stunde haben wir versucht, die Natur nur mit unseren Sinnen zu untersuchen –
also ohne Hilfsmittel und ohne Zahlen.
Dabei habt ihr verschiedene Objekte beschrieben, geschätzt und geraten. \\

Heute wollen wir herausfinden,
wie man Beobachtungen so festhalten kann,
dass sie für alle gleich verständlich und überprüfbar sind.

\vspace{0.3cm}

\subsection*{1. Rückblick auf letzte Stunde}

\subsubsection*{Welche Methoden hast du benutzt, um Informationen über die Objekte zu sammeln?}
\writeLines{4}{\textwidth}

\subsubsection*{Waren diese Methoden sinnvoll? Warum oder warum nicht?}
\writeLines{4}{\textwidth}

\newpage
\subsubsection*{Welche Methode würdest du heute lieber benutzen, um die Objekte genauer zu beschreiben?}
\writeLines{4}{\textwidth}

\vspace{0.5cm}
\subsubsection*{Wieso waren fast alle Beschreibungen ungenau? Was versteht man eigentlich unter einer ungenauen Beschreibung?}
\writeLines{2}{\textwidth}


\newpage
\subsection*{2. Neue Aufgabe: Wie viel Flüssigkeit ist in den Behältern?}

Vor dir stehen vier unterschiedlich große rechteckige Behälter, alle mit Wasser gefüllt.
Jeder sieht anders aus – aber in welchem ist am meisten Wasser?


  \subsubsection*{Was fällt dir beim bloßen \textbf{Beobachten} auf?}
  \writeLines{3}{\textwidth}

  \subsubsection*{Was könntest du tun, um \textbf{genauer herauszufinden}, wo am meisten Wasser ist?}
  \writeLines{4}{\textwidth}

  \subsubsection*{Welche Dinge kannst du an den Behältern \textbf{messen}?
  (z.\,B. Höhe, Länge, Breite, Füllstand, ...)}
  \writeLines{4}{\textwidth}




\subsection*{3. Eure Ergebnisse festhalten}



\vspace{0.1cm}
\karoBox{\textwidth}{12cm}

\subsubsection*{Was zeigt eure \textbf{Messung}? Stimmt sie mit eurer \textbf{Beobachtung} überein?}
\writeLines{3}{\textwidth}

