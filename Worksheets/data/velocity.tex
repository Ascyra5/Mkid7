\section*{Das erste vollständige Experiment}
In den Letzten Stunden haben wir immer Teile eines Experiments durchgeführt. Wir haben beobachtet, gemessen und zum Teil ausgewertet. \\
Was genau war denn nochmal der nterschied zwischen einer reinen Beobachtung und einer Messung? \\

\writeLines{2}{\textwidth}

Bis jetzt haben wir aber noch nie wirklich ein vollständiges Expriment gesehen - das holen wir aber heute nach.
Was ganz wichtig bei dem Experiment heute ist - es wird nicht darum gehen, dass wir den perfekten Weg finden, es geht ersteinmal nur darum, \textit{dass} wir einen Weg finden.
Dabei soll der Kreativität und den Ideen, um auf eine Antwort zu kommen, keinen (bzw. kaum) Grenze(n) gesetzt werden. \\

\newpage

\subsection{1. Frage, Vermutung, Hypothese}
Hier schreiben wir auf, was wir eigentlich rausfinden wollen. Welche Frage wollen wir beantworten?
versucht das so präzise und genau wie möglich zu formulieren. \\

\writeLines{4}{\textwidth}

Gibt es schon Vermutungen, was die Antwort auf die Frage sein könnte? Oder Abschätzungen? \\

\writeLines{5}{\textwidth}

\newpage
\subsection{2. Planung des Experiments}
Wie können wir die Frage beantworten? Welche Messungen müssen wir durchführen? \\
Hier ist es wichtig, dass ihr einen genauen Plan aufstellt, was ihr machen wollt. Natürlich habt ihr nur begrenzte Zeit und Materialien, aber versucht trotzdem so genau wie möglich zu planen, was ihr machen wollt. \\

\writeLines{10}{\textwidth}

\subsection{3. Durchführen des Experiments}

\newpage
\subsection{4. Messdaten}

\karoBox{\textwidth}{17cm}
\newpage
\subsection{5. Auswertung der Messdaten}
Untersuche nun, ob deine Messdaten helfen, deine Ausgangsfrage zu beantworten.\\
Falls nicht - starte wieder bei Schritt 2 oder schlimmstenfalls bei Schritt 1 ;-) \\

\karoBox{\textwidth}{15cm}
