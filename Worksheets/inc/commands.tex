% TeachingMaterials/Layouts/Worksheets/inc/commands.tex

%% =================================================================================================================
%%  empty box
%% =================================================================================================================
\newcommand{\drawbox}[2]{%
    \par
    \noindent\fbox{\parbox{#1}{\vspace{#2}\strut}}%
}
% Usage: \drawbox{<width>}{<height>}
%% =================================================================================================================



%% =================================================================================================================
%%  writing lines
%% =================================================================================================================

\newcommand{\writeLines}[2]{%
  % #1: number of lines
  % #2: width of the lines
  \noindent
  \foreach \i in {1,...,#1} {%
    \rule{#2}{0.4pt}\\[1.5em] % Linie mit Breite #2 und Abstand von 1.5em
  }
}
% Usage: \writeLines{<number_of_lines>}{<width>}
%% =================================================================================================================




%% =================================================================================================================
%%  karo box
%% =================================================================================================================

\newcommand{\karoBox}[2]{%
  % #1: width of the rectangle
  % #2: height of the rectangle
  \noindent
  \begin{tikzpicture}
    \draw[step=0.5cm, lightgray, thin] (0,0) grid (#1,#2); % Gitter mit 0.5cm Kästchen
    %\draw[lightgray, thin] (0,0) rectangle (#1,#2); % Rechteck mit Rand
  \end{tikzpicture}
}
% Usage: \karoBox{<width>}{<height>}
%% =================================================================================================================


%% =================================================================================================================
%% random number
%% =================================================================================================================

\newcommand{\randInt}[2]{%
  \number\numexpr\pdfuniformdeviate #2-#1+1 + #1\relax
}

% Usage: \randInt{<min>}{<max>}
%% =================================================================================================================